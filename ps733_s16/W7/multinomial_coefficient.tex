\documentclass[12pt]{article}

\usepackage[margin=1in]{geometry}  % set the margins to 1in on all sides
\usepackage{graphicx}              % to include figures
\usepackage{amsmath}               % great math stuff
\usepackage{amsfonts}              % for blackboard bold, etc
\usepackage{amsthm}                % better theorem environments

\usepackage{rotating} % for sideway table
\usepackage{xcolor}
\usepackage{hyperref}
\hypersetup{
    colorlinks,
    linkcolor={red!50!black},
    citecolor={blue!50!black},
    urlcolor={blue!80!black}
}
\usepackage{cleveref}

\usepackage{array,tabularx}
  
\usepackage{float}
\restylefloat{table}

% bibliography
\usepackage{natbib}
\bibpunct{(}{)}{;}{a}{}{,} % no comma between author and year

\title{Multinomial model's coefficient}
\author{Anh Le}


\begin{document}
\maketitle

We have
\begin{align}
p_{ij} &= \frac{\exp(x_i \beta_j)}{\sum_{l} \exp(x_i \beta_l)} \\
\end{align}

We want to derive the marginal effect of $x_i$ on $p_{ij}$ , so we take the derivative of $p_{ij}$ with regards to $x_{ij}$
\begin{align}
\frac{\partial p_{ij}}{\partial x_i} &= \frac{[\sum_{l} \exp(x_i \beta_l)]\exp(x_i \beta_j) \beta_j - \exp(x_i \beta_j) [\sum_l \exp(x_i \beta_l) \beta_l] }{[\sum_l \exp(x_i \beta_l)]^2} \\
&= \frac{\exp(x_i \beta_j) \beta_j}{\sum_l \exp(x_i \beta_l)} - \frac{\exp(x_i \beta_j) }{\sum_l \exp(x_i \beta_l)} \times \frac{\sum_l \exp(x_i \beta_l) \beta_l}{\sum_l \exp(x_i \beta_l)} \\
&= \frac{\exp(x_i \beta_j)}{\sum_l \exp(x_i \beta_l)} \beta_j - \frac{\exp(x_i \beta_j) }{\sum_l \exp(x_i \beta_l)} \times \sum_l\left[\frac{\exp(x_i \beta_l)}{\sum_l \exp(x_i \beta_l)}\beta_l\right] \\
&= p_{ij} \beta_j - p_{ij} \times \sum_l p_{il} \beta_l \\
&= p_{ij}(\beta_j - \sum_l p_{il} \beta_l)
\end{align}

The key point here is that even if we know the sign of $\beta_j$, we won't be able to deduce the sign of $\frac{\partial p_{ij}}{\partial x_i}$, i.e. the marginal effect of $x_j$. 

Indeed, since $p_{ij} > 0$, for $\frac{\partial p_{ij}}{\partial x_i}$ to be positive, $\beta_j - \sum_l p_{il} \beta_l$ must be positive, i.e. $\beta_j$ needs to be larger than all the other $\beta$. That is not guaranteed even if $\beta_j > 0$. So even if we know $\beta_j > 0$, we know nothing about the sign of $\frac{\partial p_{ij}}{\partial x_i}$.

\end{document}
