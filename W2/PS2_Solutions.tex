\documentclass{article}\usepackage[]{graphicx}\usepackage[]{color}
%% maxwidth is the original width if it is less than linewidth
%% otherwise use linewidth (to make sure the graphics do not exceed the margin)
\makeatletter
\def\maxwidth{ %
  \ifdim\Gin@nat@width>\linewidth
    \linewidth
  \else
    \Gin@nat@width
  \fi
}
\makeatother

\definecolor{fgcolor}{rgb}{0.345, 0.345, 0.345}
\newcommand{\hlnum}[1]{\textcolor[rgb]{0.686,0.059,0.569}{#1}}%
\newcommand{\hlstr}[1]{\textcolor[rgb]{0.192,0.494,0.8}{#1}}%
\newcommand{\hlcom}[1]{\textcolor[rgb]{0.678,0.584,0.686}{\textit{#1}}}%
\newcommand{\hlopt}[1]{\textcolor[rgb]{0,0,0}{#1}}%
\newcommand{\hlstd}[1]{\textcolor[rgb]{0.345,0.345,0.345}{#1}}%
\newcommand{\hlkwa}[1]{\textcolor[rgb]{0.161,0.373,0.58}{\textbf{#1}}}%
\newcommand{\hlkwb}[1]{\textcolor[rgb]{0.69,0.353,0.396}{#1}}%
\newcommand{\hlkwc}[1]{\textcolor[rgb]{0.333,0.667,0.333}{#1}}%
\newcommand{\hlkwd}[1]{\textcolor[rgb]{0.737,0.353,0.396}{\textbf{#1}}}%

\usepackage{framed}
\makeatletter
\newenvironment{kframe}{%
 \def\at@end@of@kframe{}%
 \ifinner\ifhmode%
  \def\at@end@of@kframe{\end{minipage}}%
  \begin{minipage}{\columnwidth}%
 \fi\fi%
 \def\FrameCommand##1{\hskip\@totalleftmargin \hskip-\fboxsep
 \colorbox{shadecolor}{##1}\hskip-\fboxsep
     % There is no \\@totalrightmargin, so:
     \hskip-\linewidth \hskip-\@totalleftmargin \hskip\columnwidth}%
 \MakeFramed {\advance\hsize-\width
   \@totalleftmargin\z@ \linewidth\hsize
   \@setminipage}}%
 {\par\unskip\endMakeFramed%
 \at@end@of@kframe}
\makeatother

\definecolor{shadecolor}{rgb}{.97, .97, .97}
\definecolor{messagecolor}{rgb}{0, 0, 0}
\definecolor{warningcolor}{rgb}{1, 0, 1}
\definecolor{errorcolor}{rgb}{1, 0, 0}
\newenvironment{knitrout}{}{} % an empty environment to be redefined in TeX

\usepackage{alltt}

\usepackage{amsmath, amsthm, amsfonts, xfrac}
\usepackage{hyperref}

\title{Pol Sci 630: Problem Set 2 Solutions - Properties of Random Variables}
\author{Prepared by: Anh Le (\href{mailto:anh.le@duke.edu}{anh.le@duke.edu})}
\date{Due Date for Grading: Friday, September 11, 2015, 10 AM (Beginning of Class)}
\IfFileExists{upquote.sty}{\usepackage{upquote}}{}
\begin{document}
\maketitle

\section*{1. Expected Value and Its Properties}

\subsection*{a.} (1/4 point) (DeGroot, p. 216) Suppose that one word is to be selected at random from the sentence `the girl put on her beautiful red hat`. If X denotes the number of letters in the word that is selected, what is the value of E(X)?

\textbf{Solution}

As the number of letters in a word, $X$ can take on following values: $x \in \{2, 3, 4, 9 \}$, with probability as follows:

\begin{align}
P(X = 2) &= \frac{1}{8} \qquad \text{(1 word (``on'') out of 8 words in the sentence)} \\
P(X = 3) &= \frac{5}{8} \\
P(X = 4) &= \frac{1}{8} \\
P(X = 9) &= \frac{1}{8}
\end{align}

Therefore,

$$E(X) = \sum_{all x_i} x_i P(X = x_i) = 3.75$$

\subsection*{b.} (2/4 point) (Degroot p. 216) Suppose that one letter is to be selected at random from
the 30 letters in the sentence given in Exercise 4. If Y
denotes the number of letters in the word in which the
selected letter appears, what is the value of E(Y)?

Hint: 1a) and 1b) force you to think carefully about the definition of expectation value. For each problem, think about what is your random variable ($X$), which values it takes on ($x \in \{?, ?, ...\}$) and with what probability ($P(X = x) = ?$)

\subsection*{c.} (1/4 point) (Degroot, p. 224) Suppose that three random variables $X_1$, $X_2$, $X_3$ are uniformly distributed on the interval [0, 1]. They are also independent. Determine the value of $E[(X_1 − 2X_2 + X_3)^2]$.

\section*{2. Variance and its properties}

For this problem, you can use the properties of expected value.

\subsection*{a.} (1/4 point) Prove that $Var(aX + b) = a^2 Var(X)$.

\subsection*{b.} (2/4 point) Prove that if two random variables are independent, the variance of the sum is the sum of the variance. In other words, if $X_1, X_2$ are independent, then

$$Var(X_1 + X_2) = Var(X_1) + Var(X_2)$$

\subsection*{c.} (1/4 point) (Degroot, p. 232) Suppose that one word is selected at random from the sentence `the girl put on her beautiful red hat`. If $X$ denotes the number of letters in the word that is selected, what is the value of $Var(X)$?


\section*{3. Binomial distribution}

(Credit to Jan) This problem is taken from Pitman (1993) Probability

Suppose a fair coin is tossed n times. Find a simple formula in terms of n and k for the following probability: $Pr(k\ heads | k-1\ heads\ or\ k\ heads)$. Please pay close attention to the formula, particularly what event is conditioned on what events. (Ch. 2.1, Problem 10 b) (p. 91)

Hint 1: Use the binomial distribution to model this.

Hint 2: Because those events are mutually exclusive, calculate the following:

$\dfrac{Pr(k\ heads)}{Pr (k\ heads) + Pr (k-1\ heads)}$

This is true because: $Pr(A | B) = \dfrac{Pr (A \cap B)}{Pr (B)}$

The intersection of events A and B in this case, $Pr (k\ heads \cap (k\ heads \cup k-1\ heads))$, reduces to $Pr(k\ heads)$ because the two events are mutually exclusive.

\section*{4. Plotting distribution}

For this problem, you'll need to Google some R techniques (e.g. side-by-side / overlapping plot). Also, label the axes and the plots accordingly.

\subsection*{a.} (1/4 point) Download a variable you are interested in, using \verb`WDI`. Plot the histogram, density plot, boxplot, and normal quantile plot.

\subsection*{b.} (1/4 point) Plot the histogram of that variable for Europe and Asia, 1) side by side (Hint: \verb`par(mfrow=c(?, ?))`), and 2) overlapping in the same plot.

\subsection*{c.} (1/4 point) Draw the scatterplot of that variable against another variable.

\subsection*{d.} (1/4 point) Label the point that represents your country (Hint: \href{https://chemicalstatistician.wordpress.com/2013/03/02/adding-labels-to-points-in-a-scatter-plot-in-r/}{Tutorial}) and color it red (Some Googling involved)
\end{document}
