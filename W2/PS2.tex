\documentclass{article}\usepackage[]{graphicx}\usepackage[]{color}
%% maxwidth is the original width if it is less than linewidth
%% otherwise use linewidth (to make sure the graphics do not exceed the margin)
\makeatletter
\def\maxwidth{ %
  \ifdim\Gin@nat@width>\linewidth
    \linewidth
  \else
    \Gin@nat@width
  \fi
}
\makeatother

\definecolor{fgcolor}{rgb}{0.345, 0.345, 0.345}
\newcommand{\hlnum}[1]{\textcolor[rgb]{0.686,0.059,0.569}{#1}}%
\newcommand{\hlstr}[1]{\textcolor[rgb]{0.192,0.494,0.8}{#1}}%
\newcommand{\hlcom}[1]{\textcolor[rgb]{0.678,0.584,0.686}{\textit{#1}}}%
\newcommand{\hlopt}[1]{\textcolor[rgb]{0,0,0}{#1}}%
\newcommand{\hlstd}[1]{\textcolor[rgb]{0.345,0.345,0.345}{#1}}%
\newcommand{\hlkwa}[1]{\textcolor[rgb]{0.161,0.373,0.58}{\textbf{#1}}}%
\newcommand{\hlkwb}[1]{\textcolor[rgb]{0.69,0.353,0.396}{#1}}%
\newcommand{\hlkwc}[1]{\textcolor[rgb]{0.333,0.667,0.333}{#1}}%
\newcommand{\hlkwd}[1]{\textcolor[rgb]{0.737,0.353,0.396}{\textbf{#1}}}%

\usepackage{framed}
\makeatletter
\newenvironment{kframe}{%
 \def\at@end@of@kframe{}%
 \ifinner\ifhmode%
  \def\at@end@of@kframe{\end{minipage}}%
  \begin{minipage}{\columnwidth}%
 \fi\fi%
 \def\FrameCommand##1{\hskip\@totalleftmargin \hskip-\fboxsep
 \colorbox{shadecolor}{##1}\hskip-\fboxsep
     % There is no \\@totalrightmargin, so:
     \hskip-\linewidth \hskip-\@totalleftmargin \hskip\columnwidth}%
 \MakeFramed {\advance\hsize-\width
   \@totalleftmargin\z@ \linewidth\hsize
   \@setminipage}}%
 {\par\unskip\endMakeFramed%
 \at@end@of@kframe}
\makeatother

\definecolor{shadecolor}{rgb}{.97, .97, .97}
\definecolor{messagecolor}{rgb}{0, 0, 0}
\definecolor{warningcolor}{rgb}{1, 0, 1}
\definecolor{errorcolor}{rgb}{1, 0, 0}
\newenvironment{knitrout}{}{} % an empty environment to be redefined in TeX

\usepackage{alltt}

\usepackage{amsmath, amsthm, amsfonts, xfrac}
\usepackage{hyperref}

\title{Pol Sci 630: Problem Set 2 - Properties of Random Variables}
\author{Prepared by: Anh Le (\href{mailto:anh.le@duke.edu}{anh.le@duke.edu})}
\date{Due Date: Tuesday, September 8, 2015, 10 AM (Beginning of Class)}
\IfFileExists{upquote.sty}{\usepackage{upquote}}{}
\begin{document}
\maketitle

Note 1: It is absolutely essential that you show all your work, including intermediary steps, and comment on your R code to earn full credit. Showing all steps and commenting on code them will also be required in future problem sets.

Note 2: Please use a *single* PDF file created through knitr to submit your answers. knitr allows you to combine R code and \LaTeX \ code in one document, meaning that you can include both the answers to R programming and math problems. Also submit the source code that generates the PDF file (i.e. either .Rnw or .Rmd files)

Note 3: Make sure that the PDF files you submit do not include any references to your identity. The grading will happen anonymously. You can submit your answer at the following website: \url{http://ps630-f15.herokuapp.com/}

\section*{1. Properties of Expected Value}

Prove the following properties, using the definition of expected values:

1. $E[aX + b] = aE[X] + b$

2. $E[X + Y] = E[X] + E[Y]$

3. If $X$ and $Y$ are independent, $E[XY] = E[X]E[Y]$

4. $Var[aX + b] = a^2 Var[X]$

\section*{2. Properties of Poisson}

Prove that a Poisson variable has equal mean and variance

\section*{3. Binomial distribution}

This problem is taken from Pitman (1993) Probability: Suppose a fair coin is tossed n times. Find a simple formula in terms of n and k for the following probability: $Pr(k\ heads | k-1\ heads\ or\ k\ heads)$. Please pay close attention to the formula, particularly what event is conditioned on what events. (Ch. 2.1, Problem 10 b) (p. 91)

Hint 1: Use the binomial distribution to model this.

Hint 2: Because those events are mutually exclusive, calculate the following:

$\dfrac{Pr(k\ heads)}{Pr (k\ heads) + Pr (k-1\ heads)}$

This is true because: $Pr(A | B) = \dfrac{Pr (A \cap B)}{Pr (B)}$

The intersection of events A and B in this case, $Pr (k\ heads \cap (k\ heads \cup k-1\ heads))$, reduces to $Pr(k\ heads)$ because the two events are mutually exclusive.

\section*{4. Plotting distribution}

For this problem, you'll need to Google some R techniques.

1. Download GDP per capita data using the \verb`WDI` package, and plot the normal quantile comparison plot to check whether GDP per capita is normally distributed.

2. Similarly, use plot to check whether log(GDP per capita) is normally distributed.

3. Plot the histograms of GDP per capita for Europe and Asia, side by side. (Hint: \verb`par(mfrow=c(?, ?))`)

4. Plot the histograms of GDP per capita for Europe and Asia, overlapping in the same plot.

\end{document}
