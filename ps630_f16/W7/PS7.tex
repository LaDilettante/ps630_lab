\documentclass[12pt]{article}
\usepackage{amsmath}
\usepackage{amssymb}
\usepackage{graphicx}
\usepackage{fullpage}
\usepackage{setspace}
\usepackage{hyperref}
\onehalfspacing



\begin{document}

\title{Pol Sci 630: Problem Set 7 - Dummy Variables and Interactions (II) and Short Research Outline}

\author{Prepared by: Jan Vogler (\href{mailto:jan.vogler@duke.edu}{jan.vogler@duke.edu})}

\date{Due Date: Wednesday, October 19th, 2016, 1.25 PM (Beginning of Class)}
 
\maketitle 

\section*{}

\paragraph{Note 1: It is absolutely essential that you show all your work, including intermediary steps, in your (mathematical) calculations and that you comment on your R code to earn full credit (you can comment on your R code both with the use of \# in the R code and in the \LaTeX \ code). Showing all steps and commenting on code will also be required in future problem sets.}

\paragraph{Note 2: Please submit a PDF file created through knitr containing all your answers to the problem set. knitr allows you to combine R code and \LaTeX \ code in one document, meaning that you can include both the answers to R programming and math problems. Also submit the source code that generates the PDF file (i.e. the .Rnw file).}

\paragraph{Note 3: Make sure that the PDF files you submit do not include any references to your identity. The grading will happen anonymously. You can submit your answer at the following website: \url{http://ps630-f15.herokuapp.com/}}




\section*{R Programming}

\subsection*{Problem 1 (4 points)}

\paragraph{Do the following in R:}

\subparagraph{a)} Load the VOTE1 dataset from last week. Estimate a linear model with the same variables that you used in problem 2 of problem set 5 and add a fixed effect for state. Display the regression results in R.

\subparagraph{b)} Then, please describe the consequences of using a fixed effect for state. Are there any differences in terms of the direction or significance of the key variables? What is different between a regular regression and one that uses fixed effects, specifically in this case?



\subsection*{Problem 2 (8 points)}

\subparagraph{a)} When explaining the vote share that the incumbent party has received, there could be an interaction between the strength of the incumbent party (prtystrA) and its expenditures (expendA). We might expect that there is a positive feedback between both as parties with a stronger organization can use the money more effectively. (Note: This expectation might or might not be true.) Please run a regression that includes an interaction term of both variables. Then display the regression results.

Note: Please \textbf{do not} include state-fixed effects in this regression.

\subparagraph{b)} Please make statements about all coefficients in your interaction model. Interpret the direction and magnitude of the marginal effects as well as the level of significance of all coefficients. How does the marginal effect of expenditures on vote share vary at different levels of party strength?

\subparagraph{c)} Please graphically show the interaction between party strength and expenditures of the incumbent party with the following plots:

\begin{enumerate}
	\item Create a marginal effects plot with confidence intervals of the influence of expenditures on vote share at different levels of party strength.
	\item Create a marginal effects plot with confidence intervals of the influence of party strength on vote share at different levels of expenditures.
	\item Create a coefficient plot in which you show the coefficients and associated confidence intervals of expenditures, party strength, and their interactions.
	\item Create a predicted values plot with confidence intervals of the influence of party strength on vote share at different levels of expenditures. Choose the 5th and 95th percentile of expenditures to show two different predicted value lines.
\end{enumerate}

\subparagraph{d)} Please answer the following questions:

\begin{enumerate}
	\item Does a lower level of party strength have a positive or negative impact on the marginal effect that expenditures have on vote share?
	\item Does a higher level of expenditures have a positive or negative impact on the marginal effect that party strength has on vote share?
	\item Considering these results, what is the substantive difference in the effects that variables have between an analysis that includes an interaction term and one that does not?
\end{enumerate}



\section*{Interactions: Math and Interpretation}

\subsection*{Problem 3 (4 points)}

Assume that you have the following linear model

$Y = 10 + (5) * X_1 + (2) * X_2 + (1) * X_1 * X_2 + u$

\subparagraph{a)} Calculate the derivative of Y with respect to $X_1$ and $X_2$.

\subparagraph{b)} Use R to plot the marginal effect of $X_1$ at different levels of $X_2$. Assume that $X_2$ is an integer that varies between -10 and 10.

\subparagraph{c)} Please interpret the marginal effect of $X_1$ based on both the derivative and the plot that you generated.

\subparagraph{d)} If there is an interaction term between $X_1$ and $X_2$ in reality, what would be the potential consequences of omitting this interaction from our model? Explain carefully.



\section*{Short Research Outline}

\subsection*{``Problem 4" (ungraded)}

\paragraph{Please write a short research outline with a minimum length of 3 paragraphs and a maximum length of 2 pages. One paragraph on each question is sufficient. Please upload your answers as a separate files to the Sakai Dropbox of the course.}

\subparagraph{a)} Write one paragraph on the background of your research idea. What is the question you want to ask? Why is it relevant? Which academic works are most closely related to the specific question you want to answer?

\subparagraph{b)} Write one paragraph on how the theoretical concepts you are interested in can be measured as variables. What is your unit of analysis? What is your dependent variable? What is your main independent variable? Which control variables would you like to include and why?

\subparagraph{c)} Write one paragraph on which datasets would be appropriate to use. Which data sets contain the data you need? How many observations are available over which time period? How difficult would it be to obtain the data?



\end{document}