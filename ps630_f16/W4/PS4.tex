\documentclass{article}\usepackage[]{graphicx}\usepackage[]{color}
%% maxwidth is the original width if it is less than linewidth
%% otherwise use linewidth (to make sure the graphics do not exceed the margin)
\makeatletter
\def\maxwidth{ %
  \ifdim\Gin@nat@width>\linewidth
    \linewidth
  \else
    \Gin@nat@width
  \fi
}
\makeatother

\definecolor{fgcolor}{rgb}{0.345, 0.345, 0.345}
\newcommand{\hlnum}[1]{\textcolor[rgb]{0.686,0.059,0.569}{#1}}%
\newcommand{\hlstr}[1]{\textcolor[rgb]{0.192,0.494,0.8}{#1}}%
\newcommand{\hlcom}[1]{\textcolor[rgb]{0.678,0.584,0.686}{\textit{#1}}}%
\newcommand{\hlopt}[1]{\textcolor[rgb]{0,0,0}{#1}}%
\newcommand{\hlstd}[1]{\textcolor[rgb]{0.345,0.345,0.345}{#1}}%
\newcommand{\hlkwa}[1]{\textcolor[rgb]{0.161,0.373,0.58}{\textbf{#1}}}%
\newcommand{\hlkwb}[1]{\textcolor[rgb]{0.69,0.353,0.396}{#1}}%
\newcommand{\hlkwc}[1]{\textcolor[rgb]{0.333,0.667,0.333}{#1}}%
\newcommand{\hlkwd}[1]{\textcolor[rgb]{0.737,0.353,0.396}{\textbf{#1}}}%
\let\hlipl\hlkwb

\usepackage{framed}
\makeatletter
\newenvironment{kframe}{%
 \def\at@end@of@kframe{}%
 \ifinner\ifhmode%
  \def\at@end@of@kframe{\end{minipage}}%
  \begin{minipage}{\columnwidth}%
 \fi\fi%
 \def\FrameCommand##1{\hskip\@totalleftmargin \hskip-\fboxsep
 \colorbox{shadecolor}{##1}\hskip-\fboxsep
     % There is no \\@totalrightmargin, so:
     \hskip-\linewidth \hskip-\@totalleftmargin \hskip\columnwidth}%
 \MakeFramed {\advance\hsize-\width
   \@totalleftmargin\z@ \linewidth\hsize
   \@setminipage}}%
 {\par\unskip\endMakeFramed%
 \at@end@of@kframe}
\makeatother

\definecolor{shadecolor}{rgb}{.97, .97, .97}
\definecolor{messagecolor}{rgb}{0, 0, 0}
\definecolor{warningcolor}{rgb}{1, 0, 1}
\definecolor{errorcolor}{rgb}{1, 0, 0}
\newenvironment{knitrout}{}{} % an empty environment to be redefined in TeX

\usepackage{alltt}

\usepackage{amsmath, amssymb}
\usepackage{graphicx}
\usepackage{hyperref}
\IfFileExists{upquote.sty}{\usepackage{upquote}}{}
\begin{document}

\title{Pol Sci 630: Problem Set 4 - Regression Model Estimation}

\author{Prepared by: Anh Le (\href{mailto:anh.le@duke.edu}{anh.le@duke.edu})}

\date{Due Date: Wed, September 28, 2015 (Beginning of Class)}

\maketitle

\section{Subset data frame}

\subsection{Download data}

Download the following data from \verb`WDI` and clean it as follows. Briefly comment on what each command does.

\begin{knitrout}
\definecolor{shadecolor}{rgb}{0.969, 0.969, 0.969}\color{fgcolor}\begin{kframe}
\begin{alltt}
\hlkwd{library}\hlstd{(WDI)}
\end{alltt}


{\ttfamily\noindent\itshape\color{messagecolor}{\#\# Loading required package: RJSONIO}}\begin{alltt}
\hlstd{d_wdi} \hlkwb{<-} \hlkwd{WDI}\hlstd{(}\hlkwc{indicator} \hlstd{=} \hlkwd{c}\hlstd{(}\hlstr{"NY.GDP.PCAP.CD"}\hlstd{,} \hlstr{"SP.DYN.IMRT.IN"}\hlstd{,} \hlstr{"SH.MED.PHYS.ZS"}\hlstd{),}
             \hlkwc{start} \hlstd{=} \hlnum{2005}\hlstd{,} \hlkwc{end} \hlstd{=} \hlnum{2010}\hlstd{,} \hlkwc{extra} \hlstd{=} \hlnum{TRUE}\hlstd{)}
\hlstd{d_wdi} \hlkwb{<-} \hlstd{d_wdi[d_wdi}\hlopt{$}\hlstd{region} \hlopt{!=} \hlstr{"Aggregates"}\hlstd{,}
       \hlkwd{c}\hlstd{(}\hlstr{"country"}\hlstd{,} \hlstr{"year"}\hlstd{,} \hlstr{"NY.GDP.PCAP.CD"}\hlstd{,} \hlstr{"SP.DYN.IMRT.IN"}\hlstd{,} \hlstr{"SH.MED.PHYS.ZS"}\hlstd{)]}
\hlkwd{colnames}\hlstd{(d_wdi)[}\hlnum{3}\hlopt{:}\hlnum{5}\hlstd{]} \hlkwb{<-} \hlkwd{c}\hlstd{(}\hlstr{'gdppc'}\hlstd{,} \hlstr{'infant_mortality'}\hlstd{,} \hlstr{'number_of_physician'}\hlstd{)}
\hlstd{d_wdi} \hlkwb{<-} \hlkwd{na.omit}\hlstd{(d_wdi)}
\end{alltt}
\end{kframe}
\end{knitrout}

\verb`infant_mortality`: number of mortality per 1000 live births

\verb`number_of_physician`: number of physician per 1000 people

\subsection{Subsetting}

Use subsetting techniques to do the following:

\begin{enumerate}
\item Show the GDP per capita of Brazil across years
\item Show the country-years where infant mortality $>$ 100 per 1000 live birth
\item Show the country-years where GDP per capita is above average
\item Show the country-years where GDP per capita is above average, but number of physician is below average
\end{enumerate}

\section{Build linear model}

\subsection{Download}

Download 2 variables of interest and build a linear model of their relationship using \verb`lm()`. Show the \verb`summary()` of results.

\subsection{Calculate the regression coefficients WITHOUT using `lm`}

Use the mathematical formula of the regression coefficients you saw in class and implement it in R. Is this result the same as the result output by `lm`?

\subsection{Model output}

Show the result with \verb`stargazer`, customizing:
\begin{itemize}
\item The labels of the independent variables (i.e. the covariate)
\item The label of the dependent variable
\item Make the model name (i.e. OLS) show up
\end{itemize}

Hint: The options to do those things are in \verb`help(stargazer)`. I have worded the task in a way that should help you find the relevant options.

\section{Calculate sum of squares and RMSE}

\begin{enumerate}
\item Extract the residuals and predicted values (fitted values) from the model object (from the linear model built above)
\item Calculate three ``sum of squares'' (TSS, RegSS, RSS)
\item Calculate the root mean square error and compare with R. (In R and stargazer, RMSE is called ``Residual standard error''.)
\end{enumerate}

Note: the data you feed to \verb`lm()` may have missing data, so R has to modify the data a little before using it. To extract the data that are actually used by \verb`lm()`, use \verb`my_model$model`. Use this data to calculate $\bar y$ in the sum of squares.

\end{document}
