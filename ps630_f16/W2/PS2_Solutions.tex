\documentclass[12pt,letter]{article}\usepackage[]{graphicx}\usepackage[]{color}
%% maxwidth is the original width if it is less than linewidth
%% otherwise use linewidth (to make sure the graphics do not exceed the margin)
\makeatletter
\def\maxwidth{ %
  \ifdim\Gin@nat@width>\linewidth
    \linewidth
  \else
    \Gin@nat@width
  \fi
}
\makeatother

\definecolor{fgcolor}{rgb}{0.345, 0.345, 0.345}
\newcommand{\hlnum}[1]{\textcolor[rgb]{0.686,0.059,0.569}{#1}}%
\newcommand{\hlstr}[1]{\textcolor[rgb]{0.192,0.494,0.8}{#1}}%
\newcommand{\hlcom}[1]{\textcolor[rgb]{0.678,0.584,0.686}{\textit{#1}}}%
\newcommand{\hlopt}[1]{\textcolor[rgb]{0,0,0}{#1}}%
\newcommand{\hlstd}[1]{\textcolor[rgb]{0.345,0.345,0.345}{#1}}%
\newcommand{\hlkwa}[1]{\textcolor[rgb]{0.161,0.373,0.58}{\textbf{#1}}}%
\newcommand{\hlkwb}[1]{\textcolor[rgb]{0.69,0.353,0.396}{#1}}%
\newcommand{\hlkwc}[1]{\textcolor[rgb]{0.333,0.667,0.333}{#1}}%
\newcommand{\hlkwd}[1]{\textcolor[rgb]{0.737,0.353,0.396}{\textbf{#1}}}%
\let\hlipl\hlkwb

\usepackage{framed}
\makeatletter
\newenvironment{kframe}{%
 \def\at@end@of@kframe{}%
 \ifinner\ifhmode%
  \def\at@end@of@kframe{\end{minipage}}%
  \begin{minipage}{\columnwidth}%
 \fi\fi%
 \def\FrameCommand##1{\hskip\@totalleftmargin \hskip-\fboxsep
 \colorbox{shadecolor}{##1}\hskip-\fboxsep
     % There is no \\@totalrightmargin, so:
     \hskip-\linewidth \hskip-\@totalleftmargin \hskip\columnwidth}%
 \MakeFramed {\advance\hsize-\width
   \@totalleftmargin\z@ \linewidth\hsize
   \@setminipage}}%
 {\par\unskip\endMakeFramed%
 \at@end@of@kframe}
\makeatother

\definecolor{shadecolor}{rgb}{.97, .97, .97}
\definecolor{messagecolor}{rgb}{0, 0, 0}
\definecolor{warningcolor}{rgb}{1, 0, 1}
\definecolor{errorcolor}{rgb}{1, 0, 0}
\newenvironment{knitrout}{}{} % an empty environment to be redefined in TeX

\usepackage{alltt}
\usepackage{amsmath}
\usepackage{amssymb}
\usepackage{graphicx}
\usepackage{fullpage}
\usepackage{setspace}
\usepackage{hyperref}
\usepackage{color}

\onehalfspacing
\IfFileExists{upquote.sty}{\usepackage{upquote}}{}
\begin{document}

\title{Pol Sci 630: Problem Set 2 - Solutions}
\author{Anh Le}

\maketitle

\textbf{\color{red} Insert your comments on the assignment that you are grading above the solution in bold and red text. For example write: "GRADER COMMENT: everything is correct! - 4/4 Points" Also briefly point out which, if any, problems were not solved correctly and what the mistake was.}

\section{Expected Value and Its Properties}

\subsection*{a.} (1/4 point) (DeGroot, p. 216) Suppose that one word is to be selected at random from the sentence `the girl put on her beautiful red hat`. If X denotes the number of letters in the word that is selected, what is the value of E(X)?

\textbf{Solution}

As the number of letters in a word, $X$ can take on following values: $x \in \{2, 3, 4, 9 \}$, with probability as follows:

\begin{align}
P(X = 2) &= \frac{1}{8} \qquad \text{(1 word (``on'') out of 8 words in the sentence)} \\
P(X = 3) &= \frac{5}{8} \\
P(X = 4) &= \frac{1}{8} \\
P(X = 9) &= \frac{1}{8}
\end{align}

Therefore,

$$E(X) = \sum_{all x_i} x_i P(X = x_i) = 3.75$$

\subsection*{b.} (2/4 point) (Degroot p. 216) Suppose that one letter is to be selected at random from
the 30 letters in the sentence given in Exercise 4. If Y
denotes the number of letters in the word in which the
selected letter appears, what is the value of E(Y)?

\textbf{Solution}

$Y$ can take on values $y \in \{2, 3, 4, 9 \}$ with probability as follows:

\begin{align}
P(Y = 2) &= \frac{2}{30} &\text{O,N} \\
P(Y = 3) &= \frac{15}{30} &\text{T,H,E, P,U,T, H,E,R, R,E,D, H,A,T} \\
P(Y = 4) &= \frac{4}{30} &\text{G,I,R,L} \\
P(Y = 9) &= \frac{9}{30} &\text{B,E,A,U,T,I,F,U,L}
\end{align}

Therefore,

$$E(Y) = \sum_{\text{all $y_i$}} y_i P(Y = y_i) = \frac{73}{15} = 4.867$$

\section{Plotting distribution}

For this problem, you'll need to Google some R techniques (e.g. side-by-side / overlapping plot). Also, label the axes and the plots accordingly.

\subsection*{a.} (1/4 point) Download a variable you are interested in, using \verb`WDI`. Plot the histogram, density plot, boxplot, and normal quantile plot.

\begin{knitrout}
\definecolor{shadecolor}{rgb}{0.969, 0.969, 0.969}\color{fgcolor}\begin{kframe}
\begin{alltt}
\hlcom{# install.packages("WDI")}
\hlkwd{library}\hlstd{(WDI)}
\end{alltt}


{\ttfamily\noindent\itshape\color{messagecolor}{\#\# Loading required package: RJSONIO}}\begin{alltt}
\hlstd{d_land} \hlkwb{<-} \hlkwd{WDI}\hlstd{(}\hlkwc{indicator} \hlstd{=} \hlkwd{c}\hlstd{(}\hlstr{"AG.LND.ARBL.ZS"}\hlstd{,} \hlstr{"NY.GDP.PCAP.KD"}\hlstd{),}
              \hlkwc{start}\hlstd{=}\hlnum{2010}\hlstd{,} \hlkwc{end}\hlstd{=}\hlnum{2010}\hlstd{,} \hlkwc{extra}\hlstd{=}\hlnum{TRUE}\hlstd{)}
\hlstd{d_land} \hlkwb{<-} \hlstd{d_land[d_land}\hlopt{$}\hlstd{region} \hlopt{!=} \hlstr{"Aggregates"}\hlstd{, ]}

\hlcom{# Rename column}
\hlkwd{colnames}\hlstd{(d_land)[}\hlkwd{colnames}\hlstd{(d_land)} \hlopt{==} \hlstr{"AG.LND.ARBL.ZS"}\hlstd{]} \hlkwb{<-} \hlstr{"arable_land_pct"}
\hlkwd{colnames}\hlstd{(d_land)[}\hlkwd{colnames}\hlstd{(d_land)} \hlopt{==} \hlstr{"NY.GDP.PCAP.KD"}\hlstd{]} \hlkwb{<-} \hlstr{"gdp_percapita"}

\hlstd{xlabel} \hlkwb{<-} \hlstr{"Arable land as % of total land"}
\hlkwd{par}\hlstd{(}\hlkwc{mfrow}\hlstd{=}\hlkwd{c}\hlstd{(}\hlnum{2}\hlstd{,} \hlnum{2}\hlstd{))}
\hlkwd{hist}\hlstd{(d_land}\hlopt{$}\hlstd{arable_land_pct,} \hlkwc{main} \hlstd{=} \hlstr{"Histogram"}\hlstd{,} \hlkwc{xlab} \hlstd{= xlabel)}
\hlkwd{plot}\hlstd{(}\hlkwd{density}\hlstd{(d_land}\hlopt{$}\hlstd{arable_land_pct,} \hlkwc{na.rm} \hlstd{=} \hlnum{TRUE}\hlstd{),} \hlkwc{main} \hlstd{=} \hlstr{"Density plot"}\hlstd{,} \hlkwc{xlab} \hlstd{= xlabel)}
\hlkwd{boxplot}\hlstd{(d_land}\hlopt{$}\hlstd{arable_land_pct,} \hlkwc{main} \hlstd{=} \hlstr{"Box plot of arable land"}\hlstd{)}
\hlkwd{qqnorm}\hlstd{(d_land}\hlopt{$}\hlstd{arable_land_pct,} \hlkwc{main} \hlstd{=} \hlstr{"Normal Q-Q plot of arable land"}\hlstd{)}
\hlkwd{qqline}\hlstd{(d_land}\hlopt{$}\hlstd{arable_land_pct)}
\end{alltt}
\end{kframe}
\includegraphics[width=\maxwidth]{figure/4a-1} 

\end{knitrout}


\subsection*{b.} (1/4 point) Plot the density plots of that variable for Europe and Asia, 1) side by side (Hint: \verb`par(mfrow=c(?, ?))`), and 2) overlapping in the same plot.

\begin{knitrout}
\definecolor{shadecolor}{rgb}{0.969, 0.969, 0.969}\color{fgcolor}\begin{kframe}
\begin{alltt}
\hlkwd{par}\hlstd{(}\hlkwc{mfrow}\hlstd{=}\hlkwd{c}\hlstd{(}\hlnum{1}\hlstd{,} \hlnum{3}\hlstd{))}
\hlstd{europe_density} \hlkwb{<-} \hlkwd{density}\hlstd{(}
  \hlstd{d_land[d_land}\hlopt{$}\hlstd{region} \hlopt{==} \hlstr{"Europe & Central Asia (all income levels)"}\hlstd{,} \hlstr{"arable_land_pct"}\hlstd{],}
  \hlkwc{na.rm}\hlstd{=}\hlnum{TRUE}\hlstd{)}
\hlstd{asia_density} \hlkwb{<-} \hlkwd{density}\hlstd{(}
  \hlstd{d_land[d_land}\hlopt{$}\hlstd{region} \hlopt{==} \hlstr{"East Asia & Pacific (all income levels)"}\hlstd{,} \hlstr{"arable_land_pct"}\hlstd{],}
  \hlkwc{na.rm}\hlstd{=}\hlnum{TRUE}\hlstd{)}
\hlkwd{plot}\hlstd{(europe_density,} \hlkwc{main} \hlstd{=} \hlstr{"Arable land (Europe)"}\hlstd{)}
\hlkwd{plot}\hlstd{(asia_density,} \hlkwc{main} \hlstd{=} \hlstr{"Arable land (Asia)"}\hlstd{)}

\hlcom{# Overlaying}
\hlkwd{plot}\hlstd{(asia_density,} \hlkwc{xlim} \hlstd{=} \hlkwd{c}\hlstd{(}\hlopt{-}\hlnum{20}\hlstd{,} \hlnum{80}\hlstd{),} \hlkwc{col}\hlstd{=}\hlstr{'red'}\hlstd{,} \hlkwc{main} \hlstd{=} \hlstr{"Asia and Europe"}\hlstd{)}
\hlkwd{lines}\hlstd{(europe_density,} \hlkwc{col}\hlstd{=}\hlstr{'blue'}\hlstd{)}
\hlkwd{legend}\hlstd{(}\hlnum{25}\hlstd{,} \hlnum{.05}\hlstd{,} \hlkwd{c}\hlstd{(}\hlstr{"Asia"}\hlstd{,} \hlstr{"Europe"}\hlstd{),}
  \hlkwc{lty}\hlstd{=}\hlkwd{c}\hlstd{(}\hlnum{1}\hlstd{,}\hlnum{1}\hlstd{),} \hlcom{# gives the legend appropriate symbols (lines)}
  \hlkwc{lwd}\hlstd{=}\hlkwd{c}\hlstd{(}\hlnum{1}\hlstd{,}\hlnum{1}\hlstd{),}\hlkwc{col}\hlstd{=}\hlkwd{c}\hlstd{(}\hlstr{"red"}\hlstd{,}\hlstr{"blue"}\hlstd{))}
\end{alltt}
\end{kframe}
\includegraphics[width=\maxwidth]{figure/4b-1} 
\begin{kframe}\begin{alltt}
\hlcom{# Tutorial for legend: http://www.r-bloggers.com/adding-a-legend-to-a-plot/}
\end{alltt}
\end{kframe}
\end{knitrout}

\subsection*{c.} (1/4 point) Draw the scatterplot of that variable against another variable.

\begin{knitrout}
\definecolor{shadecolor}{rgb}{0.969, 0.969, 0.969}\color{fgcolor}\begin{kframe}
\begin{alltt}
\hlkwd{plot}\hlstd{(d_land}\hlopt{$}\hlstd{arable_land_pct,} \hlkwd{log}\hlstd{(d_land}\hlopt{$}\hlstd{gdp_percapita),}
     \hlkwc{xlab} \hlstd{=} \hlstr{"Arable land as % of total land"}\hlstd{,}
     \hlkwc{ylab} \hlstd{=} \hlstr{"log GDP per capita (2005 USD)"}\hlstd{,}
     \hlkwc{main} \hlstd{=} \hlstr{"Arable land and GDP per capita"}\hlstd{)}
\end{alltt}
\end{kframe}
\includegraphics[width=\maxwidth]{figure/unnamed-chunk-1-1} 

\end{knitrout}


\subsection*{d.} (1/4 point) Label the point that represents your country (Hint: \href{https://chemicalstatistician.wordpress.com/2013/03/02/adding-labels-to-points-in-a-scatter-plot-in-r/}{Tutorial}) and color it red (Some Googling involved)

\begin{knitrout}
\definecolor{shadecolor}{rgb}{0.969, 0.969, 0.969}\color{fgcolor}\begin{kframe}
\begin{alltt}
\hlkwd{par}\hlstd{(}\hlkwc{mfrow}\hlstd{=}\hlkwd{c}\hlstd{(}\hlnum{1}\hlstd{,} \hlnum{1}\hlstd{))}
\hlkwd{plot}\hlstd{(}\hlkwd{log}\hlstd{(gdp_percapita)} \hlopt{~} \hlstd{arable_land_pct,}
     \hlkwc{data} \hlstd{= d_land,}
     \hlkwc{xlab} \hlstd{=} \hlstr{"Arable land as % of total land"}\hlstd{,}
     \hlkwc{ylab} \hlstd{=} \hlstr{"log GDP per capita (2005 USD)"}\hlstd{,}
     \hlkwc{main} \hlstd{=} \hlstr{"Arable land and GDP per capita"}\hlstd{)}
\hlstd{d_land_Vietnam} \hlkwb{<-} \hlstd{d_land[d_land}\hlopt{$}\hlstd{country} \hlopt{==} \hlstr{"Vietnam"}\hlstd{, ]}
\hlkwd{with}\hlstd{(d_land_Vietnam,}
     \hlkwd{text}\hlstd{(}\hlkwd{log}\hlstd{(gdp_percapita)} \hlopt{~} \hlstd{arable_land_pct,} \hlkwc{labels} \hlstd{=} \hlstr{"Vietnam"}\hlstd{,}
          \hlkwc{pos} \hlstd{=} \hlnum{3}\hlstd{,} \hlkwc{col} \hlstd{=} \hlstr{'red'}\hlstd{))}
\hlkwd{points}\hlstd{(d_land_Vietnam}\hlopt{$}\hlstd{arable_land_pct,} \hlkwd{log}\hlstd{(d_land_Vietnam}\hlopt{$}\hlstd{gdp_percapita),}
       \hlkwc{pch} \hlstd{=} \hlnum{16}\hlstd{,} \hlkwc{col} \hlstd{=} \hlstr{'red'}\hlstd{)}
\end{alltt}
\end{kframe}
\includegraphics[width=\maxwidth]{figure/unnamed-chunk-2-1} 

\end{knitrout}



\end{document}
