\documentclass[12pt]{article}
\usepackage{amsmath}
\usepackage{amssymb}
\usepackage{graphicx}
\usepackage{fullpage}
\usepackage{setspace}
\usepackage{hyperref}
\onehalfspacing



\begin{document}

\title{Pol Sci 630: Problem Set 13 - Autocorrelation, Fixed Effects, and Causal Inference Techniques}

\author{Prepared by: Jan Vogler (\href{mailto:jan.vogler@duke.edu}{jan.vogler@duke.edu})}

\date{Due Date: Tuesday, November 24th, 2015, 10 AM}
 
\maketitle 



It is absolutely essential that you show all your work, including intermediary steps, and comment on your R code to earn full credit. Showing all steps and commenting on code them will also be required in future problem sets.

Please use a *single* PDF file created through knitr to submit your answers. knitr allows you to combine R code and \LaTeX \ code in one document, meaning that you can include both the answers to R programming and math problems. Also submit the source code that generates the PDF file (i.e. either .Rnw or .Rmd files)

Make sure that the PDF files you submit do not include any references to your identity. The grading will happen anonymously. You can submit your answer at the following website: \url{http://ps630-f15.herokuapp.com/}



\section*{R Programming}

\subsection*{Problem 1 (4 points)}

\paragraph{Do the following in R:}

\subparagraph{} Load the \textit{2010 CCES\_data} that is attached to the email you received. Simulations.



\subsection*{Problem 2 (4 points)}

\paragraph{Do the following in R:}

\subparagraph{} 



\subsection*{Problem 3 (4 points)}

\paragraph{Do the following in R:}

\subparagraph{a)} 

\subparagraph{b)} 



\subsection*{Problem 4 (4 points)}

\paragraph{Do the following problems. Show every step.}

\subparagraph{a)}

\subparagraph{b)}



\end{document}